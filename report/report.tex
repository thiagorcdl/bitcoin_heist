\documentclass[10pt]{article}

\usepackage{fullpage}
\usepackage{setspace}
\usepackage{parskip}
\usepackage{titlesec}
\usepackage[section]{placeins}
\usepackage{xcolor}
\usepackage{breakcites}
\usepackage{lineno}
\usepackage{hyphenat}





\PassOptionsToPackage{hyphens}{url}
\usepackage[colorlinks = true,
            linkcolor = blue,
            urlcolor  = blue,
            citecolor = blue,
            anchorcolor = blue]{hyperref}
\usepackage{etoolbox}
\makeatletter
\patchcmd\@combinedblfloats{\box\@outputbox}{\unvbox\@outputbox}{}{%
  \errmessage{\noexpand\@combinedblfloats could not be patched}%
}%
\makeatother


\usepackage{natbib}




\renewenvironment{abstract}
  {{\bfseries\noindent{\abstractname}\par\nobreak}\footnotesize}
  {\bigskip}

\titlespacing{\section}{0pt}{*3}{*1}
\titlespacing{\subsection}{0pt}{*2}{*0.5}
\titlespacing{\subsubsection}{0pt}{*1.5}{0pt}


\usepackage{authblk}


\usepackage{graphicx}
\usepackage[space]{grffile}
\usepackage{latexsym}
\usepackage{textcomp}
\usepackage{longtable}
\usepackage{tabulary}
\usepackage{booktabs,array,multirow}
\usepackage{amsfonts,amsmath,amssymb}
\providecommand\citet{\cite}
\providecommand\citep{\cite}
\providecommand\citealt{\cite}
% You can conditionalize code for latexml or normal latex using this.
\newif\iflatexml\latexmlfalse
\providecommand{\tightlist}{\setlength{\itemsep}{0pt}\setlength{\parskip}{0pt}}%

\AtBeginDocument{\DeclareGraphicsExtensions{.pdf,.PDF,.eps,.EPS,.png,.PNG,.tif,.TIF,.jpg,.JPG,.jpeg,.JPEG}}

\usepackage[utf8]{inputenc}
\usepackage[english]{babel}










\begin{document}

\title{Bitcoin Blockchain Clustering Analysis for Ransomware Detection}



\author{Thiago R. C. de Lima}%


\vspace{-1em}



  \date{\today}


\begingroup
\let\center\flushleft
\let\endcenter\endflushleft
\maketitle
\endgroup





\selectlanguage{english}
\begin{abstract}
Bitcoin is the most popular digital currency. It~is not controlled by
any sort of central bank or government institution and is the preferred
payment method requested by cyber criminals through ransomware.~ This
type of malware encrypts a victim's files and forces them to pay a
ransom in order to regain access. In this short paper, Bitcoin
transaction data of tenyears is analyzed by generating a K-Means
clustering model, using it to predict each sample's cluster, and then
creating a~confusion matrix and evaluating the results (Rand Score).%
\end{abstract}%



\sloppy


\section*{Introduction}

{\label{489411}}

Different forms of digital currency have been proposed and implemented
over the decades~\hyperref[csl:1]{[1]}~\hyperref[csl:2]{[2]} and now the
Blockchain technology allows decentralized currencies to exist and
reliable transactions to be made, while preventing double-spending using
consensus algorithm relying on proof of work~\hyperref[csl:3]{[3]}. This is
the basis of~\emph{cryptocurrencies}, specially Bitcoin, which is often
considered the first one and has now gathered a lot of
attention~\hyperref[csl:4]{[4]}, from researchers, to investors, and, due
to its lack of regulation, criminals as well \hyperref[csl:5]{[5]}.~

Recently the internet has been targeted by a wave of new types of
malware, categorized as~\emph{ransomware.~}Ransomware\emph{~}is a type
of malware that once it successfully infects a target's computer system,
it then encrypts files and private data so that the victim cannot access
them anymore~\hyperref[csl:6]{[6]}\emph{.}~The victim is then presented
with a message explaining the situation and also containing instructions
on how to regain full access to their own system and this typically
involves paying a sum of money via Bitcoin. There are several groups of
hackers and variants, also known as~\emph{families}, of such malware,
yet their defining characteristic is requiring payment to release access
to the captured data.

On this research I will analyze several transactions gathered from the
Bitcoin Blockchain to determine whether it is possible to easily
categorize each transaction as being from a specific ransomware family
or not pertaining to any of them at all. Relevant research has been
conducted on this field in~\hyperref[csl:7]{[7]},~\hyperref[csl:8]{[8]}
and~\hyperref[csl:9]{[9]}.~

\section*{\texorpdfstring{{Development}}{Development}}

{\label{698162}}

\subsection*{\texorpdfstring{{Data}}{Data}}

{\label{472478}}

The data set used for this research was provided by Cuneyt Gurcan
Akcora, Yulia Gel and Murat Kantarcioglu and is currently available at
the UCI Machine Learning Repository~\hyperref[csl:10]{[10]}, hosted by the
University of California Irvine~\hyperref[csl:11]{[11]}. This data set
contains a total of 2916697 rows with 10 columns each, collected from
January 2009 to December 2018.~

Each row represents a transaction from the Bitcoin Blockchain and
contains the following set of
attributes:~\emph{address},~\emph{year},~\emph{day},~\emph{length},~\emph{weight},~\emph{count},~\emph{looped},~\emph{neighbors},~\emph{income}
and~\emph{label}. A full description of each of these features can be
found in~\hyperref[csl:12]{[12]}. For the purpose of this research only the
meaning of \emph{address} and \emph{label} are of importance.~

The \emph{label} attribute is used to determine the known family of
ransomware each transaction relates to, if any. There are 28 possible
families of ransomware within the dataset. The entries that are not
related to any of those families have been labeled as~\emph{white} by
\emph{Akcora et al}. It is imperative to state that this does not
necessarily mean such transactions are known to be completely unrelated
to any sort of ransomware, as there might exist families unbeknownst to
the analysts that put the data set together.

The~\emph{address}~feature represents the Bitcoin address and is not
used for the purposed of this research. The reasoning behind this
decision revolves around two issues. The first issue is this attribute
being a string value, therefore not quantifiable. The second, although
still related to the previous, pertains how such values are generated
and what kind of meaning they would bring to a clustering process. The
K-Means algorithm is based on the positioning of each sample in an
N-dimension~\emph{Euclidean Space}, where closer samples tend to have a
higher similarity. That is, the position in each axis has a meaning.
Bitcoin addresses do not hold any sort of sequential meaning since they
are randomly generated hashes.

\par\null

\subsection*{Tools}

{\label{515402}}

I have developed a script using the~\emph{Python} language,
and~\emph{SciKi-Learn~}and \emph{Pandas} packages. Python is a general
purpose programming language~\hyperref[csl:13]{[13]}~\hyperref[csl:14]{[14]} that
has gained notoriety in the data science
field~\hyperref[csl:15]{[15]}~\hyperref[csl:16]{[16]}; SciKit-Learn is tool set
for machine learning~\hyperref[csl:17]{[17]}; and Pandas is a library for
data analysis~\hyperref[csl:18]{[18]}. For this particular research, I used
SciKit-Learn for generating the K-Means model and feeding it with the
data set again to retrieve the relationship between each transaction and
the expected cluster. Then, Pandas is used to manage the data set,
cross-reference the known labels and the predicted clusters, and
evaluating the level of similarity. The source code has been published
under MIT License and can be found in my repository on
GitHub~\hyperref[csl:19]{[19]}:
\url{https://github.com/thiagorcdl/bitcoin\_heist}.

The script starts by preparing the data set: loads the CSV file
containing the data, converts the type from strings to floating point
representation when appropriate and the address column, as previously
explained. Then a copy of the data is created without the label to avoid
taking it into consideration, and the K-Means model for~\emph{k}
clusters is generated using the remaining 8
attributes:~\emph{year},~\emph{day},~\emph{length},~\emph{weight},~\emph{count},~\emph{looped},~\emph{neighbors},
and~\emph{income}. The result of this model is a set
of~\emph{k}~\emph{centroids}, which represent the center of each cluster
in an 8-dimension Euclidean Space, and are written to a CSV file. Then,
the model is used to predict the cluster for each of the entries in the
data set and the result is then used, along with the previously known
labels, to generate a Confusion Matrix, which is written to a separate
CSV file.

\par\null

\subsection*{Methodology}

{\label{153660}}

The effectiveness of the classification using K-Means is calculated by
retrieving the~\emph{Rand Index}, which measures the similarity between
how the known labels and how the predicted clusters are
plotted~\hyperref[csl:20]{[20]} in the 8-dimension space. Finally the
Adjusted Rand Index is also calculated, taking into consideration the
similarity to random model clustering \hyperref[csl:21]{[21]}.

Considering the purpose of this research is to find out whether it is
possible to segment the transactions in such way to allow the easy
categorization of whether they represent a ransomware-related
transaction or not, I decided to feed the script with two different
values for k, that is to create two different sets of clusters. The
first one is to create 29 cluster, ideally one for each family of
ransomware plus the~\emph{white} ones. Since clustering is an
unsupervised classification algorithm, the k clusters do not have a
pre-established meaning. In other words, unless there was a perfect
segregation of the 29 possible labels, it is not necessarily possibly to
know which cluster relates to which label. Thus, the second attempt was
to simply create two cluster, which a perfect fit could separate the
transactions that are known to be malicious from those that have no
evidence.

\section*{Results}

{\label{481498}}

Upon running the script for 29 clusters and 2 clusters, respectively, I
gathered the information on two confusion matrices and four Rand Index
values.

For the scenario with 29 clusters, the confusion matrix can be found in
\emph{Table~}{\ref{730387}}. The resulting Rand Index
value was~\emph{0.8550275740313222}, which relatively close to 1. This
value by itself could mean a good level of similarity. However, looking
at the confusion matrix, it is possible to note the vast majority of
samples were categorized on the very same cluster 0 and even though the
white label dominates every other cluster, it is also the majority in
cluster 0. That means that even though one could say all clusters but 0
have a high probability of not being related to ransomware, there isn't
enough confidence to actually pinpoint the ones that are. That could
mean a high rate of False Negatives . This becomes clearer when taking
into account the value of Rand Index adjusted for chance, which
yielded~\emph{-2.918553945107504}. A value below 0 is not reliable.

In \emph{Table~}{\ref{178847}} the confusion matrix for
the binary scenario is presented. The~\emph{negative} label was assigned
to all rows with original label~\emph{white}, and~\emph{positive} to the
rest, that is to the ones that are known to belong to some family of
ransomware. The Rand Index values were a bit better in this
scenario:~\emph{0.9719860133842922} and~\emph{-0.49493780907250057} for
the regular and the adjusted values, respectively, which still means an
unreliable level of similarity. The data from the confusion matrix
follows the same path as the findings of the previous scenario. Even
though there is a really high True Negative Rate~ (\emph{0.9857911192}),
the True Positive Rate is zero, which means the model was unable to
correctly guess a single suspicious transaction.

\par\null\selectlanguage{english}
\begin{table}[h!]
\centering
\tiny\begin{tabulary}{1.0\textwidth}{CCCCCCCCCCCCCCCCCCCCCCCCCCCCCC}
Labels\textbackslashPredicted & 0 & 1 & 2 & 3 & 4 & 5 & 6 & 7 & 8 & 9 & 10 & 11 & 12 & 13 & 14 & 15 & 16 & 17 & 18 & 19 & 20 & 21 & 22 & 23 & 24 & 25 & 26 & 27 & 28 \\
montrealAPT & 11 & 0 & 0 & 0 & 0 & 0 & 0 & 0 & 0 & 0 & 0 & 0 & 0 & 0 & 0 & 0 & 0 & 0 & 0 & 0 & 0 & 0 & 0 & 0 & 0 & 0 & 0 & 0 & 0 \\
montrealComradeCircle & 1 & 0 & 0 & 0 & 0 & 0 & 0 & 0 & 0 & 0 & 0 & 0 & 0 & 0 & 0 & 0 & 0 & 0 & 0 & 0 & 0 & 0 & 0 & 0 & 0 & 0 & 0 & 0 & 0 \\
montrealCryptConsole & 7 & 0 & 0 & 0 & 0 & 0 & 0 & 0 & 0 & 0 & 0 & 0 & 0 & 0 & 0 & 0 & 0 & 0 & 0 & 0 & 0 & 0 & 0 & 0 & 0 & 0 & 0 & 0 & 0 \\
montrealCryptXXX & 2419 & 0 & 0 & 0 & 0 & 0 & 0 & 0 & 0 & 0 & 0 & 0 & 0 & 0 & 0 & 0 & 0 & 0 & 0 & 0 & 0 & 0 & 0 & 0 & 0 & 0 & 0 & 0 & 0 \\
montrealCryptoLocker & 8924 & 0 & 0 & 0 & 0 & 0 & 3 & 0 & 0 & 0 & 15 & 21 & 0 & 0 & 0 & 0 & 0 & 0 & 0 & 0 & 0 & 298 & 0 & 0 & 0 & 0 & 50 & 0 & 4 \\
montrealCryptoTorLocker2015 & 54 & 0 & 0 & 0 & 0 & 0 & 0 & 0 & 0 & 0 & 0 & 0 & 0 & 0 & 0 & 0 & 0 & 0 & 0 & 0 & 0 & 1 & 0 & 0 & 0 & 0 & 0 & 0 & 0 \\
montrealDMALocker & 246 & 0 & 0 & 0 & 0 & 0 & 0 & 0 & 0 & 0 & 0 & 0 & 0 & 0 & 0 & 0 & 0 & 0 & 0 & 0 & 0 & 5 & 0 & 0 & 0 & 0 & 0 & 0 & 0 \\
montrealDMALockerv3 & 354 & 0 & 0 & 0 & 0 & 0 & 0 & 0 & 0 & 0 & 0 & 0 & 0 & 0 & 0 & 0 & 0 & 0 & 0 & 0 & 0 & 0 & 0 & 0 & 0 & 0 & 0 & 0 & 0 \\
montrealEDA2 & 6 & 0 & 0 & 0 & 0 & 0 & 0 & 0 & 0 & 0 & 0 & 0 & 0 & 0 & 0 & 0 & 0 & 0 & 0 & 0 & 0 & 0 & 0 & 0 & 0 & 0 & 0 & 0 & 0 \\
montrealFlyper & 9 & 0 & 0 & 0 & 0 & 0 & 0 & 0 & 0 & 0 & 0 & 0 & 0 & 0 & 0 & 0 & 0 & 0 & 0 & 0 & 0 & 0 & 0 & 0 & 0 & 0 & 0 & 0 & 0 \\
montrealGlobe & 32 & 0 & 0 & 0 & 0 & 0 & 0 & 0 & 0 & 0 & 0 & 0 & 0 & 0 & 0 & 0 & 0 & 0 & 0 & 0 & 0 & 0 & 0 & 0 & 0 & 0 & 0 & 0 & 0 \\
montrealGlobeImposter & 54 & 0 & 0 & 0 & 0 & 0 & 0 & 0 & 0 & 0 & 0 & 0 & 0 & 0 & 0 & 0 & 0 & 0 & 0 & 0 & 0 & 1 & 0 & 0 & 0 & 0 & 0 & 0 & 0 \\
montrealGlobev3 & 34 & 0 & 0 & 0 & 0 & 0 & 0 & 0 & 0 & 0 & 0 & 0 & 0 & 0 & 0 & 0 & 0 & 0 & 0 & 0 & 0 & 0 & 0 & 0 & 0 & 0 & 0 & 0 & 0 \\
montrealJigSaw & 4 & 0 & 0 & 0 & 0 & 0 & 0 & 0 & 0 & 0 & 0 & 0 & 0 & 0 & 0 & 0 & 0 & 0 & 0 & 0 & 0 & 0 & 0 & 0 & 0 & 0 & 0 & 0 & 0 \\
montrealNoobCrypt & 483 & 0 & 0 & 0 & 0 & 0 & 0 & 0 & 0 & 0 & 0 & 0 & 0 & 0 & 0 & 0 & 0 & 0 & 0 & 0 & 0 & 0 & 0 & 0 & 0 & 0 & 0 & 0 & 0 \\
montrealRazy & 8 & 0 & 0 & 0 & 0 & 1 & 0 & 0 & 0 & 0 & 2 & 0 & 0 & 0 & 0 & 0 & 0 & 0 & 0 & 0 & 0 & 2 & 0 & 0 & 0 & 0 & 0 & 0 & 0 \\
montrealSam & 1 & 0 & 0 & 0 & 0 & 0 & 0 & 0 & 0 & 0 & 0 & 0 & 0 & 0 & 0 & 0 & 0 & 0 & 0 & 0 & 0 & 0 & 0 & 0 & 0 & 0 & 0 & 0 & 0 \\
montrealSamSam & 62 & 0 & 0 & 0 & 0 & 0 & 0 & 0 & 0 & 0 & 0 & 0 & 0 & 0 & 0 & 0 & 0 & 0 & 0 & 0 & 0 & 0 & 0 & 0 & 0 & 0 & 0 & 0 & 0 \\
montrealVenusLocker & 7 & 0 & 0 & 0 & 0 & 0 & 0 & 0 & 0 & 0 & 0 & 0 & 0 & 0 & 0 & 0 & 0 & 0 & 0 & 0 & 0 & 0 & 0 & 0 & 0 & 0 & 0 & 0 & 0 \\
montrealWannaCry & 28 & 0 & 0 & 0 & 0 & 0 & 0 & 0 & 0 & 0 & 0 & 0 & 0 & 0 & 0 & 0 & 0 & 0 & 0 & 0 & 0 & 0 & 0 & 0 & 0 & 0 & 0 & 0 & 0 \\
montrealXLocker & 1 & 0 & 0 & 0 & 0 & 0 & 0 & 0 & 0 & 0 & 0 & 0 & 0 & 0 & 0 & 0 & 0 & 0 & 0 & 0 & 0 & 0 & 0 & 0 & 0 & 0 & 0 & 0 & 0 \\
montrealXLockerv5.0 & 7 & 0 & 0 & 0 & 0 & 0 & 0 & 0 & 0 & 0 & 0 & 0 & 0 & 0 & 0 & 0 & 0 & 0 & 0 & 0 & 0 & 0 & 0 & 0 & 0 & 0 & 0 & 0 & 0 \\
montrealXTPLocker & 8 & 0 & 0 & 0 & 0 & 0 & 0 & 0 & 0 & 0 & 0 & 0 & 0 & 0 & 0 & 0 & 0 & 0 & 0 & 0 & 0 & 0 & 0 & 0 & 0 & 0 & 0 & 0 & 0 \\
paduaCryptoWall & 12165 & 0 & 0 & 0 & 0 & 0 & 0 & 0 & 0 & 0 & 0 & 0 & 0 & 0 & 0 & 0 & 0 & 0 & 0 & 0 & 0 & 205 & 0 & 0 & 0 & 0 & 20 & 0 & 0 \\
paduaJigsaw & 2 & 0 & 0 & 0 & 0 & 0 & 0 & 0 & 0 & 0 & 0 & 0 & 0 & 0 & 0 & 0 & 0 & 0 & 0 & 0 & 0 & 0 & 0 & 0 & 0 & 0 & 0 & 0 & 0 \\
paduaKeRanger & 10 & 0 & 0 & 0 & 0 & 0 & 0 & 0 & 0 & 0 & 0 & 0 & 0 & 0 & 0 & 0 & 0 & 0 & 0 & 0 & 0 & 0 & 0 & 0 & 0 & 0 & 0 & 0 & 0 \\
princetonCerber & 9219 & 0 & 0 & 0 & 0 & 0 & 0 & 0 & 0 & 0 & 0 & 0 & 0 & 0 & 0 & 0 & 0 & 0 & 0 & 0 & 0 & 4 & 0 & 0 & 0 & 0 & 0 & 0 & 0 \\
princetonLocky & 6623 & 0 & 0 & 0 & 0 & 0 & 0 & 0 & 0 & 0 & 0 & 0 & 0 & 0 & 0 & 0 & 0 & 0 & 0 & 0 & 0 & 2 & 0 & 0 & 0 & 0 & 0 & 0 & 0 \\
white & 2690022 & 5 & 81 & 9 & 405 & 142 & 1167 & 27 & 2 & 2 & 3975 & 10940 & 10 & 97 & 26 & 635 & 2 & 3 & 9 & 78 & 193 & 131719 & 33 & 9 & 1 & 147 & 33144 & 585 & 1816 \\
\end{tabulary}
\caption{{Confusion matrix for 29 labels
{\label{730387}}%
}}
\end{table}\selectlanguage{english}
\begin{table}[h!]
\centering
\small\begin{tabulary}{1.0\textwidth}{CCC}
Labels\textbackslashPredicted & 1 & 0 \\
positive & 0 & 41413 \\
negative & 30 & 2875254 \\
\end{tabulary}
\caption{{Confusion matrix for 2 labels
{\label{178847}}%
}}
\end{table}\par\null

\subsection*{Observations}

{\label{872008}}

To summarize, in this research I utilized an extensive data set of
Bitcoin transactions to discover whether there are natural clusters with
strong correlation to ransomware occurrences. I developed a script and
used it for trying to segment the data set both for all 29 original
labels and also for a binary partitioning. I used the K-means clustering
algorithm in the attempts of finding a clear separation between
suspicious and not suspicious transactions.~

Although for both scenarios the plain Rand Index is close to 1, when
evaluating the similarity adjusting for chance the scores drop below
zero, which indicates low reliability. This leads me to conclude the
combination of features, algorithm and consequently the generated model
used do not represent a good method for classifying suspicious
transactions in the Bitcoin Blockchain.

For future work I would consider comparing the results when using the
address attribute, and also trying out different classification
algorithms.

\par\null

\selectlanguage{english}
\FloatBarrier
\section*{References}\sloppy
\phantomsection
\label{csl:1}[1]D. Chaum, ``{Blind Signatures for Untraceable Payments}'', in \textit{Advances in Cryptology}, Springer {US}, 1983, pp. 199–203.

\phantomsection
\label{csl:2}[2]C. P. Mullan, \textit{{The digital currency challenge shaping online payment systems through {US} financial regulations}}. Basingstoke: Palgrave Macmillan, 2014.

\phantomsection
\label{csl:3}[3]S. A. Back \textit{et al.}, ``{Enabling Blockchain Innovations with Pegged}'', 2014.

\phantomsection
\label{csl:4}[4]Z. Zheng, S. Xie, H.-N. Dai, X. Chen, and H. Wang, ``{Blockchain challenges and opportunities: A survey}'', \textit{International Journal of Web and Grid Services}, vol. 14, p. 352, Oct. 2018, doi: 10.1504/IJWGS.2018.095647.

\phantomsection
\label{csl:5}[5]A. Jeffries, ``{Suspected multi-million dollar Bitcoin pyramid scheme shuts down, investors revolt}''. The Verge, 2012.

\phantomsection
\label{csl:6}[6]R. Richardson and M. M. North, ``{Ransomware: Evolution, mitigation and prevention}'', \textit{International Management Review}, vol. 13, no. 1, p. 10, 2017.

\phantomsection
\label{csl:7}[7]C. G. Akcora, Y. Li, Y. R. Gel, and M. Kantarcioglu, ``{BitcoinHeist: Topological Data Analysis for Ransomware Detection on the Bitcoin Blockchain}'', \textit{arXiv preprint [Web Link]}, 2019.

\phantomsection
\label{csl:8}[8]D. Goldsmith, K. Grauer, and Y. Shmalo, ``{Analyzing hack subnetworks in the bitcoin transaction graph}'', \textit{Applied Network Science}, vol. 5, Apr. 2020, doi: 10.1007/s41109-020-00261-7.

\phantomsection
\label{csl:9}[9]R. {Rivera-Castro}, P. {Pilyugina}, and E. {Burnaev}, ``{Topological Data Analysis for Portfolio Management of Cryptocurrencies}'', in \textit{2019 International Conference on Data Mining Workshops (ICDMW)}, 2019, pp. 238–243, doi: 10.1109/ICDMW.2019.00044.

\phantomsection
\label{csl:10}[10]``{UCI Machine Learning Repository: Data Sets}''. University of California Irvine.

\phantomsection
\label{csl:11}[11]``{UCI Machine Learning Repository | re3data.org}''. University of California Irvine.

\phantomsection
\label{csl:12}[12]``{BitcoinHeistRansomwareAddress Data Set}''. University of California Irvine.

\phantomsection
\label{csl:13}[13]G. van Rossum, ``{Python tutorial}'', no. R 9526, Jan-1995.

\phantomsection
\label{csl:14}[14]``{What is Python? Powerful, intuitive programming}''. InfoWorld, 13-Nov-2019.

\phantomsection
\label{csl:15}[15]``{Python eats away at R: Top Software for Analytics, Data Science, Machine Learning in 2018: Trends and Analysis - KDnuggets}''. KDNuggets.

\phantomsection
\label{csl:16}[16]``{Python Developers Survey 2019 Results}''. JetBrains.

\phantomsection
\label{csl:17}[17]A. G\'{e}ron, ``{Hands-on machine learning with {Scikit-Learn} and {TensorFlow} concepts, tools, and techniques to build intelligentsystems}'', Apr. 2017.

\phantomsection
\label{csl:18}[18]W. Mckinney, ``{pandas: a Foundational Python Library for Data Analysis and Statistics}'', \textit{Python High Performance Science Computer}, Jan. 2011.

\phantomsection
\label{csl:19}[19]T. R. C. de Lima, ``{Bitcoin Heist Repository}''. GitHub.

\phantomsection
\label{csl:20}[20]W. M. Rand, ``{Objective Criteria for the Evaluation of Clustering Methods}'', \textit{Journal of the American Statistical Association}, vol. 66, no. 336, pp. 846–850, Dec. 1971, doi: 10.1080/01621459.1971.10482356.

\phantomsection
\label{csl:21}[21]N. Vinh, J. Epps, and J. Bailey, ``{Information Theoretic Measures for Clusterings Comparison: Variants, Properties, Normalization and Correction for Chance}'', \textit{Journal of Machine Learning Research}, vol. 11, pp. 2837–2854, Oct. 2010.

\end{document}

